\documentclass[12pt,oneside,a4paper]{article}
\usepackage[T1]{fontenc}
\usepackage[utf8]{inputenc}
\usepackage[nofoot,hdivide={2cm,*,2cm},vdivide={2cm,*,2cm}]{geometry}
\usepackage{hyperref}
\usepackage{amsmath}
\usepackage{amssymb}
\usepackage{amsthm}
\usepackage[polish]{babel}
\usepackage{listings}
\usepackage{graphicx}
\usepackage{color}
\definecolor{lbcolor}{rgb}{0.95,0.95,0.95}  

\lstset
 {
    language = python,
    columns  = flexible,
    xleftmargin = 0.5cm,
    backgroundcolor=\color{lbcolor},  
 }




\author{Michał Mikołajczyk, Wojciech Adamiec}
\title{
    \textbf{Rachunek Prawdopodobieństwa i Statystyka}\\
    \large Rozwiązanie zadania z kolokwium nr 1
}

\renewcommand{\qedsymbol}{$\blacksquare$}
\newcommand\numeq[1]%
  {\stackrel{\scriptscriptstyle \mkern-1.5mu#1\mkern-1.5mu }{=}}

\begin{document}
\maketitle

\paragraph{Treść zadania:}\

Standardowy rozkład normalny ma gęstość określoną wzorem:

\begin{equation}
\begin{aligned}
    f(x) = \frac{1}{\sqrt{2 \cdot \pi}} \text{exp}\Big(\frac{-x^2}{2}\Big), \ \ \ \ x \in R.
\end{aligned}
\end{equation}

Dla dystrybuanty otrzymujemy wyrażenie:
\begin{equation}
\begin{aligned}
    \Phi(t) = \int^{t}_{- \infty } \frac{1}{\sqrt{2 \cdot \pi}} \text{exp}\Big(\frac{-x^2}{2}\Big)\  dx,
\end{aligned}
\end{equation}

która to całka nie ma przedstawienia za pomocą funkcji elementarnych. 
Ponieważ gęstość jest funkcją parzystą zatem zachodzi związek:
\begin{equation}
\begin{aligned}
   \Phi(t) = 1 - \Phi(-t)\label{lemma}
\end{aligned}
\end{equation}

Redukujemy zatem to zadanie do obliczenia wartości poniższej całki dla ustalonego $t>0$:
\begin{equation}
\begin{aligned}
    G(t) = \int^{t}_{0} \text{exp}\Big(\frac{-x^2}{2}\Big)\  dx.\label{task}
\end{aligned}
\end{equation}

\paragraph{Rozwiązanie:}\

\begin{enumerate}
    \item 
    \begin{proof}[Dowód \eqref{lemma}:]
    $ $\newline
    Zauważmy, że: 
    \begin{equation}
    \begin{aligned}
        \int^{\infty}_{- \infty } f(x) dx = 1,\label{ass}\\
        f(x) = f(-x).
    \end{aligned}
    \end{equation}
    
    Wtedy:
    \begin{equation}
    \begin{aligned}
        \Phi(t) &= \int^{t}_{- \infty } f(x)\ dx\\
                &= \int^{\infty}_{- \infty } f(x)\ dx - \int^{\infty}_{t} f(x)\\
                &\numeq{\eqref{ass}} 1 - \int^{\infty}_{t} f(x).
    \end{aligned}
    \end{equation}
    
    Skoro $f(x) = f(-x)$, to: 
    \begin{equation}
    \begin{aligned}
    \int^{\infty}_{t} f(x)\ dx &= \int^{-t}_{- \infty } f(x)  dx 
                               &= \Phi(-t).
    \end{aligned}
    \end{equation}
    
    Zatem $\Phi(t) = 1 - \Phi(-t)$.
    
    \end{proof}
    
    \newpage
    \item Metoda obliczania całki $G(t)$ -- \eqref{task}:\\
        Korzystamy z metody Romberga. Nasze rozwiązanie bazowaliśmy na wersji znajdującej się na stronie wikipedii -- \textit{\underline{\href{https://en.wikipedia.org/wiki/Romberg\%27s_method}{Romberg's method}}}.
    
    
    
    \item Obliczanie $\Phi(t)$ na podstawie $G(t)$:
    \begin{enumerate}
        \item Jeśli $t < 0$ wówczas naszym wynikiem będzie: $1 - \Phi(-t)$
        \item Dla $t \geq 0$ mamy $\Phi(t) = \frac{1}{2} + \frac{1}{\sqrt{2\pi}} \cdot G(t)$, gdzie $\frac{1}{2}$ to pole pod wykresem naszego rozkładu w przedziale $(-\infty,0]$.
        
    \end{enumerate}
    
\end{enumerate}

\paragraph{Informacje techniczne:\\}
Do dokumentacji jest dołączony program napisany w języku \textit{Python} obliczający zarówno $\Phi(t)$ jak i samo $G(t)$. Dodatkowo dla wygody dołączamy również dokument w formacie \textit{jupyter} z tymi samymi programami oraz z kilkoma przykładami.



\paragraph{Kod źródłowy:}
$ $
\begin{lstlisting}[title={Metoda Romberga}]
def romberg(f, a, b, max_steps, acc):
  Rp = [0 for _ in range(max_steps)] # Previous row
  Rc = [0 for _ in range(max_steps)] # Current row
  h = (b-a) # Step size
  Rp[0] = (f(a) + f(b)) * h * 0.5 # First trapezoidal step

  for i in range(1, max_steps):
    h /= 2
    c = 0
    ep = 1 << (i-1) # 2^(n-1)

    for j in range(1, ep+1):
      c += f(a+(2*j-1)*h)

    Rc[0] = h*c + 0.5 * Rp[0] # R(i,0)

    for j in range(1, i+1):
      n_k = 4**j
      Rc[j] = (n_k * Rc[j-1] - Rp[j-1]) / (n_k-1) # compute R(i,j)

    if i > 1 and abs(Rp[i-1] - Rc[i]) < acc:
      return Rc[i-1]

    # swap Rn and Rc as we only need the last row
    rt = Rp
    Rp = Rc
    Rc = rt

  return Rp[max_steps-1] # return our best guess
\end{lstlisting}

\newpage

\begin{lstlisting}[title={Definicja funkcji $G(t)$}]
from math import exp
def G(t):
  def internal(x):
    return exp(-(x**2) / 2)
  return romberg(internal, 0, t, 1000, 0.00001)
\end{lstlisting}

\begin{lstlisting}[title={Definicja funkcji $\Phi(t)$}]
from math import sqrt, pi
def phi(t):
  if t < 0:
    return 1 - phi(-t)
  return 0.5 + 1/sqrt(2 * pi) * G(t)
\end{lstlisting}
\end{document}
