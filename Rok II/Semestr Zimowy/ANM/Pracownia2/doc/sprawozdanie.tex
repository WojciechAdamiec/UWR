\documentclass[12pt,a4paper]{article}
\usepackage{hyperref}
\usepackage{graphicx}
\usepackage{amsmath}
\graphicspath{{./images/}}
\hypersetup{
    colorlinks,
    citecolor=black,
    filecolor=black,
    linkcolor=black,
    urlcolor=black
}
\usepackage[T1]{fontenc}
\usepackage[utf8]{inputenc}
\usepackage[polish]{babel}
\usepackage{indentfirst}
\usepackage[margin=0.8in]{geometry}

\usepackage[dvipsnames]{xcolor}

\newcommand{\re}[1]{\textcolor{red}{#1}}

\newcommand{\bl}[1]{\textcolor{blue}{#1}}

\newcommand{\gr}[1]{\textcolor{Plum}{#1}}


\begin{document}
\author{Wojciech Adamiec, 310064}
\title{
	\textbf{Pracownia 2}\\
	\large Analiza Numeryczna (M)\\
	\large Prowadzący: Witold Karczewski\\
	\large Zadanie P2.9	
}

\maketitle

\tableofcontents
\section{Wstęp}


\subsection{Treść Zadania}
Wykorzystaj aproksymację średniokwadratową, aby dopasować funkcję postaci $y=ab^x$ do danych:\\
\begin{center}
	\begin{tabular}{|c|c|c|c|c|c|} \hline
	$x$ & 1     & 2     & 3     & ... & m   \\ \hline
	$y$ & $y_1$ & $y_2$ & $y_3$ & ... & $y_m$   \\ \hline
	\end{tabular}
\end{center}
Opracowaną metodę przetestuj na wybranych przykładach.


\subsection{Plan Działania}
Zgodnie z poleceniem, do rozwiązania naszego problemu zastosujemy aproksymację średniokwadratową. Nie możemy jednak zastosować jej wprost, bowiem funkcja postaci $y=ab^x$ nie jest wielomianem. Aby rozwiązać ten problem, dokonamy kilku przekształceń:\\
\[ y=ab^x\]
Logarytmujemy obustronnie równianie:
\[ ln(y)=ln(ab^x)\]
Korzystamy z własności logarytmów:
\[ ln(y)=ln(a)+ln(b^x)\]
\[ ln(y)=ln(a)+x\cdot ln(b)\]
Zauważamy, że mamy już teraz postać wielomianu - w naszym przypadku funkcji liniowej. Użyjmy teraz zmiennych pomocniczych: $Y=ln(y)$, $A=ln(a)$ i $B=ln(b)$. Dostajemy wówczas:
\[Y=A+x\cdot B\]
Teraz możemy już bez problemu zastosować aproksymację średniokwadratową. Szukamy minimum wyrażenia:
\[||Y - (A + x\cdot B))||^2\]
W tym celu musimy rozwiązać równanie:
\[
\begin{bmatrix}
\langle 1, 1 \rangle & \langle 1, x \rangle\\
\langle x, 1 \rangle & \langle x, x \rangle
\end{bmatrix}
\cdot
\begin{bmatrix}
A\\
B
\end{bmatrix}
=
\begin{bmatrix}
\langle Y, 1 \rangle\\
\langle Y, x \rangle
\end{bmatrix}
\]
W naszym przypadku przyjmujemy dyskretną definicję iloczynu skalarnego, co upraszcza nam równanie do:
\[
\begin{bmatrix}
m & \sum_{k=1}^{m} x_k \\
\sum_{k=1}^{m} x_k & \sum_{k=1}^{m} x_k^2
\end{bmatrix}
\cdot
\begin{bmatrix}
A\\
B
\end{bmatrix}
=
\begin{bmatrix}
\sum_{k=1}^{m} Y_k \\
\sum_{k=1}^{m} Y_k x_k
\end{bmatrix}
\]
Teraz pozostaje nam rozwiązać układ 2 równań liniowych, aby otrzymać $A$ i $B$. Końcowym krokiem będzie wyliczenie $a$ i $b$ zgodnie ze wzorami:
\[ a=e^A\]
\[ b=e^B\]
Zatem nasz plan działania sprowadza się do:
\begin{itemize}
\item Obliczenia $Y_k=ln(y_k)$ dla $k=1, 2, 3, ..., m$
\item Obliczenia $\sum_{k=1}^{m} x_k$ oraz $\sum_{k=1}^{m} x_k^2$
\item Obliczenia $\sum_{k=1}^{m} Y_k$ oraz $\sum_{k=1}^{m} Y_k x_k$
\item Rozwiązania układu 2 równań liniowych
\item Obliczeniu $a$ i $b$
\end{itemize}
Do zadania dołączony jest program, który wykonuje wszystkie te czynności i od razu zwraca wartości $a$ i $b$.

\subsection{Sprawy Techniczne}
Wszystkie obliczenia numeryczne zostały wykonane w języku \textit{Julia} w wersji \textit{v1.2.0}, w podwójnej precyzji - \textit{binary64} zgodnie ze standardem \textit{IEEE 754}, którego opis można znaleźć np. tu: \url{https://en.wikipedia.org/wiki/IEEE_754}

\section{Rozwiązanie Zadania}
W treści zadania nie mamy żadnych konkretnych danych, zatem naszą metodę zastosujemy do kilku własnoręcznie wymyślonych przykładów. Będziemy je tworzyć w taki sposób, że najpierw wybierzemy pewną funkcję postaci $y=ab^x$, następnie dla $m$ różnych wartości $x$ odczytamy i zapiszemy odpowiadające wartości $y$. Na koniec dokonamy lekkiego \textit{zaburzenia} wartości $x$ i $y$, tak aby zasymulować rzeczywistą sytuację.

\subsection{Przykład 1}
Dla sprawdzenia poprawności przedstawionej metody zacznijmy od prostej funkcji $y=2^x$. W tym przykładzie wyjątkowo nie dokonamy zaburzenia wartości, aby upewnić się, że nasza metoda jest poprawna. Jako dane weźmiemy:
\begin{center}
	\begin{tabular}{|c|c|c|c|c|c|} \hline
	$x$ & 1 & 2 & 3 & 4 & 5   \\ \hline
	$y$ & 2 & 4 & 8 & 16& 32  \\ \hline
	\end{tabular}
\end{center}
Czyli $m=5$.\\
Stosując nasz algorytm otrzymujemy:\\
$a = 1$ oraz $b = 2.0000000000000004$, zatem otrzymaliśmy w niemal idealnym przybliżeniu funkcję wejściową. Pozwala to nabrać intuicji, że nasz algorytm faktycznie działa.
\subsection{Przykład 2}
Weźmy teraz funkcję $y = 2 \cdot 3^x$. Zaburzymy nieznacznie wartości - nasze dane będą takie:\\ 
\begin{center}
	\begin{tabular}{|c|c|c|c|c|c|} \hline
	$x$ & 1 & 2 & 3 & 4 & 5   \\ \hline
	$y$ & 5.87 & 18.05 & 54.1 & 162.6 & 487.1  \\ \hline
	\end{tabular}
\end{center}
Stosując nasz algorytm otrzymujemy:\\
$a \approx 1.9662069$ i $b \approx 3.014812$, w poszczególnych punktach daje nam to:\\
\begin{center}
	\begin{tabular}{|c|c|c|c|c|c|} \hline
	$y$  & 5.87 & 18.05 & 54.1 & 162.6 & 487.1   \\ \hline
	$y'$ & 5.93 & 17.87 & 53.88 & 162.43 & 489.70  \\ \hline
	$|\Delta|$ & 0.06 & 0.18 & 0.22 & 0.17 & 2.6  \\ \hline
	\end{tabular}
\end{center}


\subsection{Przykład 3}
Sprawdźmy teraz jakość algorytmu w zależności od ilości badanych punktów. W tym celu weźmy teraz funkcję $y = 7.4284 \cdot 1.5672^x$ dla $m=3$ i $m=5$.\\ 
Zacznijmy od $m=3$:\\
\begin{center}
	\begin{tabular}{|c|c|c|c|} \hline
	$x$ & 12.5677 & 23.2512  & 44.1234 \\ \hline
	$y$ & 2104.556 & 255724.457 & 3023070299.5619  \\ \hline
	\end{tabular}
\end{center}
Otrzymujemy:\\
$a \approx 7.428399896$ i $b \approx 1.5672$
Zauważamy, że już dla $m=3$ wynik jest niemal perfekcyjny, wynika to z tego, że nie zaburzyliśmy naszych danych. Zróbmy zatem lekkie zaburzenie:\\
\begin{center}
	\begin{tabular}{|c|c|c|c|} \hline
	$x$ & 12.56 & 23.25  & 44.12 \\ \hline
	$y$ & 2102 & 255725 & 3023070315  \\ \hline
	\end{tabular}
\end{center}
Otrzymujemy wówczas:\\
$a \approx 7.441683$ i $b \approx 1.567179$\\
Dołóżmy dwa punktu do naszych danych (lekko zaburzone, $m=5$), sprawdzimy jak zmieni się nasze dopasowanie:\\
\begin{center}
	\begin{tabular}{|c|c|c|c|c|c|} \hline
	$x$ & 12.56 & 16.2 & 23.25 & 25.5 & 44.12 \\ \hline
	$y$ & 2102 & 10760 & 255725 & 702365 & 3023070315  \\ \hline
	\end{tabular}
\end{center}
Otrzymujemy:\\
$a \approx 7.43292$ i $b \approx 1.567213$\\
Otrzymane dopasowanie jest trochę lepsze, ale nie jesteśmy w stanie stwierdzić, czy to ze względu na dodatkowe punkty czy przez \textit{inne} zaburzenie dwóch dodanych punktów. Łatwo jednak dostrzec, że różnica jest niewielka.
\section{Podsumowanie}
Metoda aproksymacji średniokwadratowej pozwala szybko i z dużą dokładnością dopasowywać funkcję postaci $y=ab^x$ do odpowiednich danych. Przykład $1$ pokazał, że metoda pozwala niemal perfekcyjnie odtworzyć funkcję wykładniczą na bazie zaledwie kilku punktów. Pozostałe przykłady pozwalają nabrać przekonania, że metoda nie tylko działa, ale również daje stosunkowo dokładny wynik.

\end{document}

